\documentclass[11pt, a4paper, sans]{moderncv}

\moderncvstyle{classic}
\moderncvcolor{blue}

\setlength{\hintscolumnwidth}{2cm}

\usepackage[utf8]{inputenc}
\usepackage[scale=0.8]{geometry}
\usepackage{helvet}
\usepackage[english]{babel}
\usepackage{tikz}

\DeclareUnicodeCharacter{2605}{\starfull}
\DeclareUnicodeCharacter{2606}{\starempty}

\name{Quentin}{FOURNIER}
\title{Curriculum Vitae}
\address{82, Boulevard Carnot}{59\,800 LILLE}{FRANCE}
\phone[mobile]{(+33)\,06~68~07~85~41}
\email{xeheros@gmail.com}
\homepage{xeheros.com}
\social[linkedin]{quentinfournier}
\social[twitter]{xeheros}
\social[github]{xeheros}

\photo[128pt][0.4pt]{profil.jpg}

\begin{document}

\makecvtitle{}

\section{Education}

\cventry{2013--2015}{Master 2 Informatique}{Université Lille 1}{Villeneuve d'Ascq}{\textit{Spécialité Image, Vision, Interaction}}{}
\cventry{2012--2013}{Licence Informatique}{Université Lille 1}{Villeneuve d'Ascq}{\textit{Global programmer}}{}
\cventry{2010--2012}{DUT Informatique}{Institut Universitaire et Technologique}{Reims}{\textit{Global programmer}}{}

\section{Work Experience}

\cventry{2018--2019}{Game Programmer Generalist}{YS Interactive}{Valenciennes}
            {recherche et développement sur des nano-membranes de Silicium.}
{
\begin{itemize}
\item  Mise en œuvre et procédés en salle blanche ;
\item intégration et caractérisation des membranes MEMS d'épaisseur nanométrique
  \begin{itemize}
  \item AFM ;
  \item  Vibromètre laser ;
  \item MEB.
  \end{itemize}
\end{itemize}}

\cventry{2015--2017}{Game Programmer Generalist}{ENIGAMI}{Tourcoing}
            {recherche et développement sur des nano-membranes de Silicium.}
{
\begin{itemize}
\item  Mise en œuvre et procédés en salle blanche ;
\item intégration et caractérisation des membranes MEMS d'épaisseur nanométrique
  \begin{itemize}
  \item AFM ;
  \item  Vibromètre laser ;
  \item MEB.
  \end{itemize}
\end{itemize}}

\section{Personal Projects}

\section{Skills}

\cvdoubleitem{}{\begin{tikzpicture}
\node [anchor=west] at (.1,.8) {Unreal Engine 4};
\draw [fill=gray] (0,0) rectangle (5,.5);
\draw [fill={rgb:red,1;green,2;blue,3}] (0,0) rectangle (4.5,.5);
\end{tikzpicture}}{}{\begin{tikzpicture}
\node [anchor=west] at (.1,.8) {Unity 3D};
\draw [fill=gray] (0,0) rectangle (5,.5);
\draw [fill={rgb:red,1;green,2;blue,3}] (0,0) rectangle (4,.5);
\end{tikzpicture}}

\cvdoubleitem{}{\begin{tikzpicture}
\node [anchor=west] at (.1,.8) {C\#};
\draw [fill=gray] (0,0) rectangle (5,.5);
\draw [fill={rgb:red,1;green,2;blue,3}] (0,0) rectangle (4,.5);
\end{tikzpicture}}{}{\begin{tikzpicture}
\node [anchor=west] at (.1,.8) {C++};
\draw [fill=gray] (0,0) rectangle (5,.5);
\draw [fill={rgb:red,1;green,2;blue,3}] (0,0) rectangle (3,.5);
\end{tikzpicture}}
\cvitem{}{\begin{tikzpicture}
\node [anchor=west] at (.1,.8) {\LaTeX};
\draw [fill=gray] (0,0) rectangle (5,.5);
\draw [fill={rgb:red,1;green,2;blue,3}] (0,0) rectangle (2,.5);
\end{tikzpicture}}

\section{Languages}

\cvitemwithcomment{French}{Native}{}
\cvitemwithcomment{English}{Read, Written, Spoken}{}

\section{Hobbies}

\cvitem{Game Design}{I like making game concept.}

\cvitem{Photography}{Shooting my holidays.}

\end{document}